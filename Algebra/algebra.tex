\documentclass[10pt,a4paper]{article}
\usepackage[utf8]{inputenc}
\usepackage{amsmath}
\usepackage{amsfonts}
\usepackage{amssymb}
\begin{document}
\section{Gruppi}

\subsection{Definizioni base}

\textbf{Def} un \textit{magma} è un insieme M in cui è definita una singola operazione binaria. L'unico assioma soddisfatto dall'operazione è quello di chiusura.\\\\
\textbf{Def} un magma associativo si dice \textit{semigruppo}\\
\textbf{Esempio} $(\mathbb{Z}^+,+)$, $(\mathbb{N},\times)$\\\\
\textbf{Def} un \textit{monoide} è una terna $(M,*,e)$ dove M è un insieme chiuso rispetto a * che è un'operazione associativa con elemento neutro $e$. Un monoide quindi è un semigruppo con elemento neutro\\
\textbf{Esempio} $(\mathbb{Z},\times)$\\\\
\textbf{Def} un \textit{gruppo} è una terna $(G,*,1)$ dove $(G,*,1)$ è un monoide in cui ogni elemento è invertibile
\subsection{Sottogruppo normale}
Def: Sia H sottogruppo di G, H si dice normale\\ 
$H \lhd G$ se $ghg^{-1} \in H$ $\forall g\in G ,h \in H$\\\\
Esempio: $K \lhd H \lhd G$ non è detto che $K \lhd G$ (vd esempio wiki)

\subsection{Gruppi abelianizzati e commutatori}
Def. $[g,h] :=g^{-1}h^{-1}gh$ commutatore\\
$[H,K] = $\{$[h,k] : h\in H,k\in K$\} per $H,K \subset G$\\\\
Def: il gruppo $[G,G]$ viene detto sottogruppo dei commutatori\\
Oss. un elemento di $[G,G]$ non è per forza delle forma $[g,h]$\\\\
Lemma: $[G,G] \lhd G$\\
Oss. G è abeliano $\iff [G,G] = $\{1\}\\\\
Lemma: sia N un sottogruppo normale di G, allora\\
$G/N $ è abeliano $\iff [G,G] \lhd N$ ovvero il sottogruppo dei commutatori \\è il piu' piccolo sottogruppo normale di G\\\\
$Ab(G) = G/[G,G]$ abelianizzato\\
$G \simeq G^{'} \Rightarrow Ab(G) \simeq Ab(G^{'})$ ma non viceversa
\newpage
\subsection{Gruppo risolubile}

Def. Un gruppo G è detto risolubile se esiste una sequenza di sottogruppi\\
$$ G=G_1 \supset G_2 \supset ... \supset G_m =\{1\} $$ tale che\\
- $G_{k+1} \lhd G_k$\\
- $ G_k/G_{k+1} $ è abeliano\\\\
Esempio: $S_3$ è risolubile\\\\
Def: sia G un gruppo, poniamo\\
- $G^{(1)} = G$\\
- $G^{(k+1)}=[G^{(k)},G^{(k)}]$\\\\
La serie
$$ G = G^{(1)} \supset G^{(2)} \supset ... \supset G^{(k)}$$
viene detta serie derivata.\\\\
Thm: sia G un gruppo. Allora G è risolubile $\iff \exists m>0$ t.c. $G^{(m)}=\{1\}$
\subsection{Gruppo ciclico}
È un gruppo che puo' essere generato da un unico elemento.\\
Sia G ciclico. Se $|G | = n$ finito allora $G \simeq \mathbb{Z}/n\mathbb{Z}$ altrimenti $ G \simeq \mathbb{Z}$\\
$g^i$ genera $\iff (i,n)=1$\\
Oss. un gruppoo ciclico è abeliano\\
\textbf{Prop.} Ogni sottogruppo ed ogni gruppo quoziente di un gruppo ciclico è ciclico.
\\\\
G = \{$g^n : g \in \mathbb{Z}$\} notazione moltiplicativa\\
G = \{$ng : n\in\mathbb{Z}$\} notazione additiva\\\\
\textbf{Thm} ogni sottogruppo finito $G$ del gruppo moltiplicativo di un campo $E$ è ciclico.\\\\
\textbf{Prop} sia G un gruppo ciclico finito, $a \in G$ allora\\ $x^n = a$ in G ha soluzioni $\iff a^{\frac{|G|}{(n,|G|)}} = 1$\\\\
\textbf{Oss} se G è un gruppo finito e $(n,|G|)=1$ allora $x^n=a$ ha soluzione in G $\forall a \in G$

\subsubsection{Radici n-esime dell'unita'}
$R_n = $\{$\alpha \in \mathbb{C} : \alpha^n =  1$\} radici n-esime dell'unita'\\
Def n-esimo polinomio ciclotomico
$$\Phi_n = \prod_{\zeta \in RPU}(x-\zeta)$$
Lemma:
$$\prod_{d|n}\Phi_d = x^n-1$$
Lemma: $\Phi_n$ è un polinomio monico di $\mathbb{Z}[X]$e ha grado $\varphi(n)$\\
Thm: $\Phi_n$ è irriducibile in $\mathbb{Q}[X]$\\
Def: sia $F$ un campo e $w \in F$ una radici primitiva n-esima dell'unita',\\
allora $F(w)/F$ è detta n-esima estensione ciclotomica di $F$\\
Def: un'estensione di Galois $E/F$ è detta ciclica\\
 se $Gal(E/F)$ è un gruppo ciclico


\subsection{Gruppo di torsione}
Un gruppo di torsione o gruppo periodico è un gruppo in cui ogni elemento ha ordine finito.Tutti i gruppi finiti sono di torsione.\\
Il concetto di gruppo di torsione non va confuso con quello di gruppo ciclico:\\$(\mathbb{Z},+)$ è ciclico senza essere di torsione.\\\\
Tor(G) = \{ $g\in G: g^n=1$ \} notazione moltiplicativa\\
Tor(G) = \{ $g\in G: ng=0$ \} notazione additiva\\\\
Sia $\varphi : G \to G^{'}$ isomorfismo allora\\
$\varphi(Tor(G)) = Tor(G{'})$

\subsection{Gruppo diedrale}
Gli elementi base del gruppo sono le rotazioni del poligono pari all'n-esima parte dell'angolo giro, e la riflessione attorno ad un asse di simmetria del poligono. Esistono in tutto n rotazioni possibili e n assi di simmetria per un poligono di n lati, per cui il gruppo diedrale corrispondente è formato da 2n elementi.\\
Esempio: quadrato\\
$<x,y | x^4=y^2=(xy)^2=1>$

\subsection{Esempi}
\textbf{Gruppi comuni}\\
$ (\mathbb{Z},+) , (\mathbb{Q}^*,\times)$\\
$S_n$ gruppo delle permutazioni,non è abeliano\\\\
\textbf{Il gruppo simmetrico $S_n$}\\
Sia $S_n$ l'insieme di tutte le mappe biiettive da $$\pi : \{1,2,..,n\} \to \{1,2,..,n\}$$
$\pi \in S_n$ è detta permutazione di $\{1,2,..,n\}$\\
Oss. $| S_n | = n!$\\
Lemma: $S_n$ è generato da $(1,2),(1,3),..,(1,n)$\\
Lemma: $S_n$ è generato da $(1,2)$ e $(1,2,..,n)$\\\\
Def: una coppia $(i,j)$ è detta inversione della permutazione $\pi$ se $\pi(i) > \pi(j)$.Denotiamo con $\varphi(n)$ il numero di inversioni di una permutazione.\\\\
Lemma: $\varphi(\pi \sigma) = \varphi(\pi) +\varphi(\sigma)$ mod 2\\
Prop. in ogni rappresentazione di $\pi$ come prodotto di 2-cicli, il numero di 2-cicli sara' sempre pari o sempre dispari.\\\\
Oss. il prodotto di due permutazioni pari è ancora pari e l'inverso di una permutazione pari è ancora pari.\\\\
Def: l'insieme delle permutazioni pari forma un sottogruppo di $S_n$ detto gruppo alterno $A_n$\\\\
Lemma: $A_n$ è generato dai 3-cicli $(i,j,k)$ per $i,j,k$ distinti\\
Prop. $A_n \lhd S_n$, $S_n/A_n \simeq \{1,-1\}$, segue che 
$|A_n| = \frac{1}{2}|S_n| = \frac{n!}{2}$\\\\
Thm: $[S_n,S_n] = A_n$\\\\
\textbf{Gruppo generale lineare}\\
$ GL_n(K)$ gruppo generale lineare: matrici invertibili di dimensione n a valori in K\\
$SL_n(K)$ gruppo lineare speciale: sottogruppo delle matrici avente determinante uguale a 1\\
Oss. non sono commutativi per $n>1$\\
Oss. $SL_n(K) \lhd GL_n(K)$ (sottogruppo normale, vd thm Binet)\\\\
$ O_n(K) = $\{$A \in GL_n(K) | A^TA = AA^T = I$\} gruppo ortogonale\\
$SO_n(K)$ gruppo ortogonale speciale (gruppo delle rotazioni dello spazio)\\\\
\textbf{Gruppo di Galois}\\
Sia E/F estensione di campi\\
$Gal(E/F) =$\{$\sigma: E \to E : \sigma$ automorfismo t.c. $\sigma(a) = a$ $ \forall a\in F$\}\\\\
\textbf{Gruppo fondamentale}\\
$\pi(X,x_0)$ è un gruppo rispetto al cammino prodotto di classi di equivalenza di cappi omotopi con punto base $x_0$\\\\
$H_q(C) := Z_q(C) / B_q(C)$ q-esimo gruppo di omologia\\
dove $Z_q(C) := \ker \delta_q $ e $ B_q(C) := $Im$\delta_{q+1}$
\newpage
\section{Teoremi sui gruppi e congruenze}

\textbf{Def.} funzione di Eulero
$\varphi (n) = \#U(\mathbb{Z}/n\mathbb{Z}) = \{a \in \mathbb{Z}: 0 \leq a < n,(a,n)=1\}$\\
\textbf{Lemma:} se p è primo $\varphi(p)=p-1$\\
\textbf{Prop.} $\varphi(p^k) = p^k-p^{k-1}$\\
\textbf{Thm.} se $(m,n)=1$ allora $\varphi(mn)= \varphi(m)\varphi(n)$\\\\
\textbf{Prop.} $\sum_{d|n}\varphi(d) =n$\\\\
\textbf{Teorema di Lagrange}\\
Sia G un gruppo finito e H un suo sottogruppo, $|G:H|$ l'indice di H in G (il numero di classi laterali di H in G) allora $|G| = |H| |G:H|$\\
Corollario: il periodo di $a \in G$ divide l'ordine di G\\\\
\textbf{Teorema di Eulero Fermat}\\
Se $(a,n)=1$ allora $a^{\varphi(n)} \equiv 1 \mod n$\\\\
\textbf{Piccolo teorema di Fermat}\\
$a^{p} \equiv a$ (mod p)\\\\
\textbf{Teorema di Wilson}\\
$(p-1)! \equiv -1 \mod p$\\\\
\textbf{Prop.} se $p \equiv 1 \mod 4$ allora $-1$ è un quadrato mod p ovvero $\exists x$ t.c. $x^2 \equiv -1 \mod p$\\\\
\textbf{Prop.} se $p=2^n+1$ è un primo allora 3 è una radice primitiva mod p\\
\subsection{Gruppo degli elementi invertibili}
\textbf{Prop.} $(1+ap)^{p^{\beta -2}} \equiv 1+ap^{p^{\beta -1}} \mod p^\beta$\\\\ 
\textbf{Lemma:}  sia $p > 2$ primo e $a$ un intero non multiplo di p.\\
Allora $1+ap$ ha ordine $p^{\alpha-1}$ mod $p^\alpha$\\\\
\textbf{Lemma:} se $a$ non è multiplo di 2 la classe resto $1+4a \mod 2^\alpha$ ha ordine $2^{\alpha-2}$\\
Oss. è la versione del lemma precedente per $p=2$\\\\
\textbf{Prop} il gruppo $U(\mathbb{Z}/p^\alpha\mathbb{Z})$ è ciclico per $p>2$ primo\\\\
\textbf{Teorema} $U(\mathbb{Z}/m\mathbb{Z})$ è ciclico se e solo se $ m=2,4,p^k,2p^k $\\\\
\textbf{Prop.} Supponiamo esista una radice primitiva $\mod m$ con $(a,m)=1$\\
Allora $a$ è una potenza n-esima (cioè $x^n \equiv a \mod m$ ha soluzione) $\iff a^{\frac{\varphi(m)}{(n,\varphi(m))}}=1$\\
Oss. deriva dalla proposizione più generale sui gruppi ciclici\\\\
\textbf{Prop.}sia $p>2$ primo, $p \nmid a,p \nmid n$, se $x^n \equiv a \mod p$ è risolubile allora anche\\
$x^n \equiv \mod p^e$ è risolubile $\forall e \geq 1$
\subsection{Legge di reciprocità quadratica}
\textbf{Def.} di resto quadratico e simbolo di Legendre\\\\
\textbf{Prop. di Eulero} $a^{\frac{p-1}{2}}\equiv (\frac{a}{p}) \mod p$\\
\textbf{Corollario} $(\frac{ab}{p}) = (\frac{a}{p})(\frac{b}{p})$,  
 $(\frac{1}{p}) = 1$,  $(\frac{-1}{p}) = (-1)^{\frac{p-1}{2}}$
\\\\
\textbf{Legge accessoria} $$\bigg (\frac{2}{p}\bigg) \equiv (-1)^{\frac{p^2-1}{8}}$$\\
cioè 1 se $p \equiv \pm 1 \mod 8, -1$ se  $p \equiv \pm3 \mod 8$\\\\
\textbf{Teorema (Legge di reciprocità quadratica)} $$\bigg ( \frac{p}{q} \bigg) \equiv (-1)^{\frac{(p-1)(q-1)}{4}} \bigg ( \frac{q}{p} \bigg)$$
Che si può riformulare come: $(\frac{p}{q}) (\frac{q}{p}) = -1$ se $p,q \equiv -1 \mod 4$, 1 altrimenti\\\\
\textbf{Def} somma di Gauss
$$ g_a= \sum_{t=0}^{p-1} \big ( \frac{t}{p} \big ) \zeta^{at}$$
Oss di fatto è il prodotto fra un carattere moltiplicativo e uno additivo\\\\
\textbf{Lemma}
$$ \sum_{t=0}^{p-1} \big ( \frac{t}{p} \big ) = 0$$
\textbf{Lemma} se $p \nmid a $
$$ \sum_{t=0}^{p-1} \zeta^{at} = 0$$
Oss sono ricnonducibili al caso più generale:\\\\
\textbf{Lemma} $\sum_{g\in G} \chi (g) = 0$ se $\chi$ non è il  carattere banale\\\\
\textbf{Prop}  $g_a = (\frac{a}{p} ) g_1$\\\\
\textbf{Prop} $g_1^2 = (-1)^{\frac{p-1}{2}}p$
\newpage
\section{Numeri di Mersenne,Fermat e Carmichael}
\textbf{Def} $M_p = 2^p-1$ è detto numero di Mersenne\\
\textbf{Def} $F_n := 2^{2^n}+1$ è detto numero di Fermat\\
\textbf{Def} p si dice \textit{elite prime} se è una radice primitiva modulo tutti i primi di Fermat salvo un numero finito\\\\
\textbf{Teorema di Pepin} $F_n$ è primo $\iff 3^{\frac{F_n -1}{2}}\equiv -1\ \mod F_n$

\subsection{Divisori della forma $b^n \pm 1$}
\textbf{Prop} $b,n>1$ interi. Se un primo $p | b^n-1$ allora o $p|b^d-1$ per un divisore proprio d di n o $p\equiv 1 \mod n$\\\\
\textbf{Prop} $b,n>1$ interi. Se un primo $p>2$,$p | b^n+1$ allora o $p|b^d+1$ per un divisore proprio d di n con $\frac{n}{d}$ dispari o $p\equiv 1 \mod 2n$\\\\
\textbf{Prop} un divisore primo p di $F_k$ con $k>1$ soddsfa $p\equiv 1 \mod 2^{k+2}$
\newpage
\section{Anelli}
\subsection{The ring of integers}
The rings of integers of number fields may be divided in several classes:\\
- Quelli che non sono PID e quindi non sono domini Euclidei come $\mathbb{Q}[\sqrt{-5}]$\\
- Quelli che sono PID e non sono domini Euclidei come $\mathbb{Q}[\sqrt{-19}]$\\
- Quelli che sono Euclidei ma non norm-Euclidean come $\mathbb{Q}[\sqrt{69}]$\\
- Quelli che sono norm-Euclidean come gli interi di Gauss (gli interi di $\mathbb{Q}[\sqrt{-1}]$\\\\
The norm-Euclidean quadratic fields have been fully classified, they are $\mathbb{Q}[\sqrt{d}]$ where d is:\\
-11, -7, -3, -2, -1, 2, 3, 5, 6, 7, 11, 13, 17, 19, 21, 29, 33, 37, 41, 57, 73
\subsection{Gli interi algebrici}
\textbf{Prop.} i numeri algebrici formano un campo\\
\textbf{Prop.} gli interi algebrici formano un anello\\
\textbf{Prop.} un numero complesso è un intero algebrico sse il suo polinomio minimo su Q ha coefficienti interi.\\\\
\textbf{Thm} se $D \equiv 2,3$ (mod 4) allora $\mathbb{Q}(\sqrt D) = \mathbb{Z}[\sqrt D]$\\
se $D \equiv 1$ (mod 4) allora $\mathbb{Q}(\sqrt D) = \mathbb{Z}[\frac{1+ \sqrt D}{2}]$
\subsection{Ideale}
\textbf{Def.} sia A un anello,$I \subset A$ si dice \textbf{ideale} di A se\\
1) I è sottogruppo di $(A,+,\times)$\\
2) $x \in I$ e $a \in A$ allora $ax,xa \in I$\\\\
\textbf{Def.} se un ideale è generato da un solo elemento diciamo che è principale\\\\
\textbf{Oss.} Un ideale che sia contemporaneamente destro e sinistro si dice ideale \textbf{bilatero}. Nel caso particolare in cui A sia un anello commutativo le nozioni date coincidono e parliamo semplicemente di ideale.\\\\
\textbf{Def.} un ideale si dice \textbf{proprio} se è un sottoinsieme proprio di A cioè non coincide con A.\\\\
\textbf{Def.} Un ideale proprio è un ideale \textbf{massimale} se non è contenuto strettamente in nessun altro ideale proprio\\
\textbf{Oss.} Gli ideali massimali sono pertanto caratterizzati dalla proprietà di essere contenuti solamente in due ideali: l'intero anello e l'ideale massimale stesso\\\\
\textbf{Def.} un ideale proprio è detto ideale \textbf{primo} se $\forall ab \in I$ allora a o b appartengono a I.
\textbf{Proprietà}
L'anello quoziente $A/I$ è un dominio $\iff I$ è un ideale primo\\
L'anello quoziente $A/I$ è un campo $\iff I$ è un ideale massimale
\textbf{Operazioni sugli ideali}
$I+J = \{a+b|a \in I, b \in J\}$\\
$IJ = \{a_1b_1+..+a_n b_n| a_i \in I,b_i \in J,i=1,..,n$ per $n=1,2,..\}$\\
Osservazioni: \\
$IJ \subset I \cap J$\\ 
$I \cup J \subset I+J$\\
$I \cap J$ è ancora un ideale mentre $I \cup J$ non sempre
\subsection{Domini}
Un dominio è un anello commutativo con unità in cui vale la legge di annullamenteo del prodotto\\
Su un dominio è definita una funzione Norma\\
\textbf{Oss.} $\varepsilon$ è invertibile $\implies N(\varepsilon)=1$\\
se vale anche l'altra implicazione la norma di dice \textit{speciale}
\textbf{Dominio Euclideo}
È un dominio dotato di una norma in cui è possibile fare la divisione con resto.\\\\
\textbf{Definizione (1)}\\
Un dominio R è euclideo se $\exists d:R \to \mathbb{N}$ t.c.
$\forall a,b, \in R, b\neq 0$ $\exists q,r,\in \mathbb{R}$ t.c.\\
$a= bq+r$\\
$d(r) < d(b)$\\
\textbf{Definizione (2)}\\
Come (1) però con $d(a) \leq d(ab)$\\
Oss. dato un dominio euclideo R si dimostra che se ne può modificare la norma d in modo che soddisfi (2)\\
\textbf{Definizione (3)}\\
Come (1) però con $ d(ab) = d(a)d(b) $\\
\textbf{Definizione (4)}\\
Limitatamente a un number ring (anello degli interi algebrici) in un number field ci si può chiedere se vale (3) con d la norma ordinaria cioè $N_{\mathbb{K}/{\mathbb{Q}}}$\\
Se questo vale si dice che R è \textit{norm-Euclidean}\\\\
\textbf{Lemma:} la norma di un dominio euclideo è speciale\\
\textbf{Oss.} in un dominio euclideo primo = irriducibile\\
\textbf{Thm} ogni dominio euclideo è un PID\\
\textbf{Thm} ogni dominio euclideo è un UFD
\subsection{PID}
A si dice PID (Principal ideal domain) se è un dominio è ogni ideale di A è principale.\\
\textbf{Thm} PID $\implies$ UFD\\\\
\textbf{Def} un PID si dice \textbf{Noetheriano} se soddisfa la ACC (condizione sulle catene ascendenti) ovvero ogni catena ascendente di ideali\\
$$ (a_1) \subseteq (a_2) \subseteq ...$$
è stazionaria cioè esiste un indice $k$ t.c. $(a_k)=(a_{k+1})=...$\\\\
\textbf{Esempio} PID che non è un dominio euclideo: $\mathbb{Z}[\frac{1}{2}(1+\sqrt {-19})]$
\subsection{UFD}
\textbf{Def.} un dominio si dice a fattorizzazione unica se ogni elemento non nullo e non invertibile di D\\
1) si scrive come prodotto di irriducibili\\
2) i fattori irriducibili di due fattorizzazioni sono gli stessi con le stesse molteplicità e a meno di associati\\
\textbf{Esempio:} UFD che non è un PID\\
1) $K[X,Y]$: l'ideale generato da $(x,y)$ non è principale\\
2) $Z[X]$: l'ideale generato da (2,x) non è principale\\\\
\textbf{Prop.} se D è un UFD $ \implies D[X]$ è un UFD
\newpage
\section{Campi}
\subsection{Norma e Traccia}
\textbf{Def.} the (field) norm maps elements of a larger field into a subfield\\
Sia $E/F$ un'estensione di Galois di grado finito, allora la norma e la traccia sono definite rispettivamente come
$$N(\alpha) = \prod_{\sigma \in G} \sigma(\alpha)$$
$$ Tr(\alpha) = \sum_{\sigma \in G} \sigma(\alpha)$$
\textbf{Prop.} sia G = Gal(E/F), e $H = \{\sigma \in G | \sigma(\alpha) = \alpha\}$ lo stabilizzatore di $\alpha$ e\\ $f = x^m+a_{m-1}x^{m-1}+...+a_0$ sia il polinomio minimo di $\alpha$ su F.Allora:\\
$$N(\alpha) = (-1)^{|G|}a_o^{|H|}$$
$$Tr(\alpha) = - |H|a_{m-1}$$
\subsection{Estensioni}
Sia $E/F$ un'estensione algebrica\\\\
\textbf{Estensione di Galois} sia $E^G := \{a \in E | \sigma(a) =a \forall \sigma \in G\} $ dove $G=Gal(E/F)$\\$E^G$ = F \\\\
\textbf{Estensione normale} se ogni polinomio irriducibile in $F[X]$ che ha una radice in $E$ ha tutte le radici in $E$.\\\\
\textbf{Estensione separabile} se il polinomio minimo di ogni $\alpha \in F$ è separabile \\\\
\textbf{Estensione ciclotomica} $E \supset F $ campo di spezzamento di $x^n-1$\\\\
\textbf{Estensione ciclica} se il suo gruppo di Galois è ciclico.\\\\
\textbf{Def} si dice \textit{torre radicale} una successione di estensioni $F=F_1 \subset F_2 \subset ... \subset F_m$ t.c. $F_{i+1} = F_i(\alpha_i)$ con $\alpha_i^{n_i} \in F_i$ \\\\
\textbf{Estensione radicale} se esiste una torre radicale \\
$F=F_1 \subset F_2 \subset ... \subset F_m = E$
\newpage
\section{Polinomi}
\subsection{Definizioni}
\textbf{Polinomio minimo:}\\
$E \supset F, \alpha \in E,f \in F[X]$\\
$f$ monico e di grado minimo t.c. $f(\alpha)=0$\\\\
\textbf{Polinomio irriducibile:}\\quando i suoi unici divisori sono 1 e lui stesso \\\\
\textbf{Polinomio separabile:} \\Ogni fattore irriducibile ha radici distinte nel campo di spezzamento\\\\
\textbf{Polinomio ciclotomico:}\\
Il polinomio minimo di $\zeta_n$ su $\mathbb{Q}$ dove $\zeta_n = e^{\frac{2\pi i}{n}}$\\\\
\textbf{Polinomio primitivo:}\\
$f\in \mathbb{Z}[X]$ si dice primitivo se il massimo comun divisiore di tutti i coefficienti è 1.\\\\
\textbf{Polinomio caratteristico:}\\
$p_A(x) := \det(A-xI_n)$\\
dove A è una matrice quadrata di dimensione $n$ a coefficienti in un campo $\mathbb{K}$
\subsection{Proposizioni e teoremi}
\textbf{Lemma di Gauss:} il prodotto di due polinomi primitivi è primitivo. \\
\textbf{Corollario:} se un polinomio è irriducibile in $\mathbb{Z}[X]$ allora è irriducibile anche in $\mathbb{Q}[X]$\\\\
\textbf{Prop.} se $p(x) \in \mathbb{Z}[X]$ e $p(0),p(1)$ sono entrambi dispari $\implies p(x)$ non ha soluzioni intere (p.31 libro)
\newpage
\section{Morfismi}
\textbf{Def} un morfismo è un'applicazione $f: A \to B$ che conserva le operazioni\\\\
\textbf{Isomorfismo} morfismo biiettivo\\\\
\textbf{Omomorfismo} morfismo tra due strutture algebriche dello stesso tipo \\\\
\textbf{Endomorfismo} è un omomorfismo con $A=B$\\\\
\textbf{Automorfismo} è un endomorfismo biiettivo, ovvero un isomorfismo con $A=B$\\
\subsection{Esulando dall'algebra}
\textbf{Omeomorfismo} è una funzione fra spazi topologici continua, biunivoca e con inversa continua\\\\
\textbf{Diffeomorfismo} è una funzione tra due varietà differenziabili con la proprietà di essere differenziabile, invertibile e di avere l'inversa differenziabile.
\end{document}